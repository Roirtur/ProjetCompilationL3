\documentclass{report}
\usepackage[utf8]{inputenc}
\usepackage{graphicx}
\usepackage{blindtext}
\usepackage{hyperref}
\usepackage{subcaption}
\usepackage{fancyhdr} % Import fancyhdr package
\usepackage{lipsum} % Generate random filler text
\usepackage{titlesec} % Load titlesec package
\usepackage{geometry} % Access extensive page dimension controls
\usepackage{listings}

\geometry{
  margin = 1in % Sets equal margins for all sides (modify as desired)
}

\titleformat{\chapter}[hang]
  {\normalfont\huge\bfseries}
  {\thechapter\hspace{1em}}
  {0pt}
  {\bfseries}


\pagestyle{fancy} % Enable custom headers and footers
\fancyhf{} % Reset header and footer fields
\fancyhead[RE,RO]{Nathan Vanbeselaere - Arthur Macdonald} % Top-right header field
\renewcommand{\footrulewidth}{0pt} % Remove separator line

\title{\textbf{\Huge Rapport de Projet Compilation }}
\author{Nathan Vanbeselaere - Arthur Macdonald}
\date{Avril 2024}

\begin{document}

\maketitle
\newpage

\tableofcontents
\newpage

\chapter*{Réponses aux questions}
    \section*{Questions sur le Lexer}

        Nous n'avons pas rencotrées de difficultées pour le Lexer.\\
        Pour traiter les commentaires sur plusieurs lignes, on crée un nouvel état dans lequel on incrémente le buffer à chaque retour a la ligne et qui s'arrete seulement lorsqu'il rencotre les charactères "*/". 

    \section*{Questions sur le Parser}
blabla  
    \section*{Questions passe renommage}
    1. Pourquoi n'est-il pas gênant que dans deux blocs disjoints (pas l'un dans l'autre) un même nom soit utilisé pour des variables locales à ces blocs ?\\ 
    
    \quad Il n'est pas gênant que dans deux blocs disjoints un même nom soit utilisé pour des variables locales à ces blocs car les variables locales sont déclarées dans des environnements différents. Ainsi, les variables locales de deux blocs disjoints n'ont pas de lien entre elles et ne peuvent pas être confondues.\\

    2. Dans le programme suivant (il s'agit du programme renaming.pix, qui n'a pas d'intérêt particulier autre que pour cet exercice), indiquez comment le renommage des variables sera effectué :\\
    
    \begin{lstlisting}[language=C, basicstyle=\ttfamily]
    Program after analysis:
    Arguments <
      Int : x;
      Real : y
      >
    $<
        Set(x, (2*x));
        Real : x#1;
        Set(x#1, (2.*y));
        Int : y#1;
        Set(y#1, Floor(x#1));
        $<
            Int : x#2;
            Set(x#2, (2*y#1));
            Int : y#2;
            Set(y#2, (2*x#2));
            
        >$;
        Set(y#1, (2*y#1));
        $<
            Coord : x#2;
            Set(x#2, Coord(y#1, y#1));
            Color : y#2;
            Set(y#2, Color(x#2.X, x#2.X, x#2.X));
            Draw_pixel(Pixel(x#2, y#2));
            
        >$;
        Set(x#1, (2.2*x#1));
        
    >$
    \end{lstlisting}

\chapter*{Difficultées rencotrées}
    Blabla


\end{document}