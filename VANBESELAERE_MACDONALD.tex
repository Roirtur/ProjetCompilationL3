\documentclass{report}
\usepackage[utf8]{inputenc}
\usepackage{graphicx}
\usepackage{blindtext}
\usepackage{hyperref}
\usepackage{subcaption}
\usepackage{fancyhdr} % Import fancyhdr package
\usepackage{lipsum} % Generate random filler text
\usepackage{titlesec} % Load titlesec package
\usepackage{geometry} % Access extensive page dimension controls

\geometry{
  margin = 1in % Sets equal margins for all sides (modify as desired)
}

\titleformat{\chapter}[hang]
  {\normalfont\huge\bfseries}
  {\thechapter\hspace{1em}}
  {0pt}
  {\bfseries}


\pagestyle{fancy} % Enable custom headers and footers
\fancyhf{} % Reset header and footer fields
\fancyhead[RE,RO]{Nathan Vanbeselaere - Arthur Macdonald} % Top-right header field
\renewcommand{\footrulewidth}{0pt} % Remove separator line

\title{\textbf{\Huge Rapport de Projet Compilation }}
\author{Nathan Vanbeselaere - Arthur Macdonald}
\date{Avril 2024}

\begin{document}

\maketitle
\newpage

\tableofcontents
\newpage

\chapter*{Réponses aux questions}
    \section*{Questions sur le Lexer}

        Nous n'avons pas rencotrées de difficultées pour le Lexer.\\
        Pour traiter les commentaires sur plusieurs lignes, on crée un nouvel état dans lequel on incrémente le buffer à chaque retour a la ligne et qui s'arrete seulement lorsqu'il rencotre les charactères "*/". 

    \section*{Questions sur le Parser}

    Questions `a faire apparaˆıtre dans le document Dans votre document, vous pr´ecise-
    rez les difficult´es rencontr´ees si votre parseur n’accepte pas tout ce qui est d´ecrit ici.
    Vous pouvez bien ´evidemment jouer avec menhir pour r´epondre `a ces questions.
    1. On consid`ere la s´equence suivante : If #expr# If #expr# #stmt# Else #stmt#
    On consid`ere pour cette question que #expr# et #stmt# sont des terminaux (i.e., on
    ne cherchera pas `a les ≪´etendre≫).
    9
    (a) Donnez les deux arbres de d´erivations possible de cette s´equence dans la gram-
    maire d´ecrite plus haut.
    (b) Donnez l’´etat de l’automate LR0 o`u apparaˆıt le conflit qui montre l’existence
    de ces deux arbres.
    (c) Quelle annotation permet d’obtenir l’arbre coh´erent avec la priorit´e d´ecrite dans
    ce document ?
    (d) Pouvez-vous via une annotation obtenir le comportement inverse ? Pourquoi ?
    2. Choisissez un conflit shift-reduce possible dans votre grammaire sans annotation (qui
    n’est pas celui de la question pr´ec´edente), et expliquez quelles annotations de priorit´e
    vous avez mis pour le r´esoudre, et son effet sur les arbres accept´es (quels arbres sont
    privil´egi´es, lesquels sont ignor´es). Vous illustrerez un exemple o`u ce conflit pourrait
    arriver et les deux arbres mis en jeu via une s´equence de tokens

    \section*{Questions passe renommage}
    Questions `a faire apparaˆıtre dans le document Dans votre document, vous pr´ecise-
    rez les difficult´es rencontr´ees si votre renommage n’effectue pas tout ce qui est d´ecrit ici.
    1. Pourquoi n’est-il pas gˆenant que dans deux blocs disjoints (pas l’un dans l’autre) un
    mˆeme nom soit utilis´e pour des variables locales `a ces blocs ?
    2. Dans le programme suivant (il s’agit du programme renaming.pix, qui n’a pas
    d’int´erˆet particulier autre que pour cet exercice), indiquez comment le renommage
    des variables sera effectu´e :


\chapter*{Difficultées rencotrées}
    Blabla


\end{document}